RBO is a protocol for a powerful sender and tiny (battery powered) receivers.
The sender repeatedly broadcasts a sequence of many (possibly millions) of messages,
while each receiver is interested in reception of a specific message within this
sequence.
The transmission is arranged so that the receiver can wake up in arbitrary moment and find
 the nearest transmission of its searched message.  
Even if it does not know the position of the message in the sequence,
it needs only
to receive a small number of other messages to locate it properly.
Thus it can save energy by keeping the radio switched off most of the time.
We show that bit-reversal permutation has ``recursive bisection properties''
and, as a consequence,
RBO can be implemented very efficiently 
with only constant number of $\lceil\log_2 n\rceil$-bit variables,
where $n$ is the total number of messages in the sequence. 
The total time of required receptions is $O(\log n)$
(at most $2\lceil\log_2 n\rceil+3$ in the model with perfect synchronization).
The basic procedure of RBO 
(computation of the time slot for the next required reception)
requires only $O(\log^3 n)$ bit-wise operations.
We propose implementation mechanisms 
for realistic model (with imperfect synchronization),
for operating systems (such as e.g. TinyOS).
