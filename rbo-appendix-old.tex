\appendix

\section{Implementation examples of some RBO functions}
\label{}
This appendix contains excerpts of the Java implementation of the
prototype of the RBO Protocol.
Complete source codes of this implementation can
be found at \cite{RBO-WWW}.
\subsection{Basic functions (module Rbo)}
\label{Rbo-Java}
Reversing bits:
\begin{verbatim}
public static int revBits(int k, int x) // reverse of k lowest bits
{
    int y= (x&1);
    for(int i=1; i<k; i++)
        {
            y= y<<1;
            x= x>>1;
            y= (y | (x&1));
        }
    return y;
}
\end{verbatim}
Shift on the level of $x$:
\begin{verbatim}
public static int shift(int k, int x) 
// shift on the level of x (and step on the next level)
{
    int mask = (1<<k)-1;
    while( (x&mask) != 0 ) mask = mask >> 1;
    return mask^(mask>>1);
}

\end{verbatim}
Level of $y$:
\begin{verbatim}
public static int lv(int y)  // level of y-value y in the tree of bs_k
{
    int l=0;
    while(y!=0) 
        {
            y=(y>>1);
            l++;
        }
    return l;
}
\end{verbatim}
Minimal $y$ in the $revBits_k(\{r_1,\ldots r_2\})$:
\begin{verbatim}
public static int minRevBits(int k, int r1, int r2)
// min(revBits([r1,r2])) // we assume: 0<=r1<=r2< 2^k
{
    int x=0; // root
    int s=shift(k, x);
    while(x<r1 || r2<x)
        {
            if(x<r1)
                // x=x+shift(k, x);
                x=x+s;
            else
                // x=x-shift(k, x);
                x=x-s;
            s=s>>1;
        }
    return revBits(k, x);
}
\end{verbatim}
Maximal \ $y$ in the $revBits_k(\{r_1,\ldots r_2\})$:
\begin{verbatim}
public static int maxRevBits(int k, int r1, int r2)
// max(revBits([r1,r2])) // we assume: 0<=r1<=r2< 2^k
{
int mask= (1<<k)-1;
return mask ^ minRevBits(k, r2^ mask, r1^mask );
}
\end{verbatim}


First time slot $y$ after $t$ (modulo $2^k$), 
such that $revBits_k(y\bmod 2^k) \in [r_1, r_2]$.
Polylogarithmic time version of the $nextSlotIn$ computation:
\begin{verbatim}
public static int plogNextSlotIn(int k, int t, int r1, int r2)
// (t+min{d>0 : r1<= revBits( (t+d)mod 2^k   ) <=r2}) mod 2^k 
// we assume 0<=r1<=r2< 2^k 
// iterative version
{
int rec=0; // compensates for the (removed) tail recursion
int tFirst, tLast, l, minL, maxL, aboveL, tFirstL, tLastL;
int shift, stepmask;

while(true)
    {

        if(r1<r2) // test if r1 or r2 can be removed
            {
                int r=revBits(k, t);
                if(r==r1) r1++; // possible reduction to singleton
                else if(r==r2) r2--; // possible reduction to singleton
            }

        if(r1==r2) // [r1,r2] is a singleton -  no choice
            return rec+revBits(k,r1); 

        tFirst=minRevBits(k, r1,r2); // first slot in [r1,r2]

        if(t < tFirst)  // we are before the entrace to [r1,r2] in this round
            return rec+(tFirst);

        tLast= maxRevBits(k, r1,r2); // last slot in [r1,r2]

        if(tLast <= t) // wait till the entrance to [r1,r2] in the next round 
            return rec+(tFirst); 

        // here: t<tLast

        // find min{ l>= lv(t): level l intersects [r1,r2] } 
        // (it must exist since: t<= tLast) 
        l=lv(t);
        shift=(1<<(k-l));
        stepmask= ~((shift<<1)-1);
        minL=((r1+shift-1)&stepmask); // "&stepmask" instead of division
        maxL=((r2-shift)&stepmask);   // "&stepmask" instead of division
        while(minL>maxL) // [r1,r2] does not intersect level l
            {
                l++;
                shift=shift>>1;
                stepmask=stepmask>>1;
                minL=((r1+shift-1)&stepmask);
                maxL=((r2-shift)&stepmask);
            }
        // [minL+shift, maxL+shift] is the minimal interval that 
        // contains intersection of level l with [r1,r2] 
        minL= minL>>(k-l+1); // now the division
        maxL= maxL>>(k-l+1); // now the division
        // [minL, maxL] are now the coresponding ranks within the level l
        tFirstL=minRevBits(l-1, minL, maxL); // entrance to [minL, maxL] in the level l 
        aboveL= 1<<(l-1); // number of nodes above the level l 
        if(t< aboveL+tFirstL) // next slot is the first within level l		       
            return rec+ aboveL+tFirstL; 


        // here: l=lv(t) and t>= aboveL+tFirstL
        tLastL=maxRevBits(l-1, minL,maxL);
        if(t>= aboveL+tLastL)
            {
                // here: l=lv(t) and t>=aboveL+tLastL and l<k 
                // (since t<maxRevBits(k, r1,r2)).
                // next slot in [r1,r2] after t is 
                // the first one within some  level after lv(t)

                l++;
                shift=shift>>1;
                stepmask=stepmask>>1;
                minL=((r1+shift-1)&stepmask);
                maxL=((r2-shift)&stepmask);

                while(minL>maxL)
                    {
                        l++;
                        shift=shift>>1;
                        stepmask=stepmask>>1;
                        minL=((r1+shift-1)&stepmask);
                        maxL=((r2-shift)&stepmask);
                    }

                minL= minL>>(k-l+1);
                maxL= maxL>>(k-l+1);

                aboveL= (1<< l-1);
                return rec+ aboveL+minRevBits(l-1, minL, maxL);
            }

        // next slot after t is within level l=lv(t)
        // return aboveL+ nextSlotIn(l-1, t-aboveL, minL, maxL); 
        // RECURSION !
        rec=rec+aboveL; // accumulates for tail recursion
        // settle the new values of parametres for recursion:
        k=l-1;
        t=t-aboveL;
        r1=minL;
        r2=maxL;
    }
}
\end{verbatim}

The following are the versions useful for special cases:
\begin{verbatim}
public static int naiveNextSlotIn(int k, int t, int r1, int r2)
// (t+min{d>0 : r1<= revBits( (t+d)mod 2^k   ) <=r2}) mod 2^k 
// we assume 0<=r1<=r2< 2^k 
{
int mask=(1<<k)-1; // 2^k-1 
t=((t+1) & mask);  // (t+1) mod 2^k
int r=revBits(k, t);
while(r<r1 || r2<r)
    {
        t=((t+1) & mask);
        r=revBits(k, t);
    }
return t;
}
\end{verbatim}
\begin{verbatim}
public static int reverseNextSlotIn(int k, int t, int r1, int r2)
// (t+min{d>0 : r1<= revBits( (t+d)mod 2^k   ) <=r2}) mod 2^k 
// we assume 0<=r1<=r2< 2^k 
{
int n=(1<<k);
int t1=revBits(k, r1);
int globalMin=t1;
int minAfter=(t1>t)? t1: n;
for(int r=r1+1; r<=r2; r++)
    {
        t1=revBits(k, r);
        if(t1<globalMin) globalMin=t1;
        if(t1>t && t1<minAfter) minAfter=t1;
    }
if(minAfter<n) return minAfter;
else return globalMin; 
}
\end{verbatim}


We can use the following heuristics to select the 
actual procedure for computing $nextSlotIn$.
\begin{verbatim}
public static int nextSlotIn(int k, int t, int r1, int r2)
// (t+min{d>0 : r1<= revBits( (t+d)mod 2^k   ) <=r2}) mod 2^k 
// we assume 0<=r1<=r2< 2^k 
{
    if(r1==r2) return revBits(k,r1);
    int length=r2-r1;
    int lengthReverse= (1<<k)/length;
    if(length<150 || lengthReverse<150)
        if(lengthReverse<=length-20)
            return naiveNextSlotIn(k,t,r1,r2);
        else
            return reverseNextSlotIn(k,t,r1,r2);
    else
        return plogNextSlotIn(k,t,r1,r2);
}
\end{verbatim}

